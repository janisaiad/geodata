\begin{abstract}
\vskip-.2cm
Optimal transport (OT) and maximum mean discrepancies (MMD) are now routinely used in machine learning to compare probability measures. %The former induces a rich Riemannian structure on the space of probability measures, the latter a comparatively simpler and flat Hilbertian structure that is also easier to handle numerically. 
We focus in this paper on \emph{Sinkhorn divergences} (SDs), a regularized variant of OT distances which can interpolate, depending on the regularization strength $\varepsilon$, between OT ($\varepsilon=0$) and MMD ($\varepsilon=\infty$). Although the tradeoff induced by that regularization is now well understood computationally (OT, SDs and MMD require respectively $O(n^3\log n)$, $O(n^2)$ and $n^2$ operations given a sample size $n$), much less is known in terms of their \emph{sample complexity}, namely the gap between these quantities, when evaluated using finite samples \emph{vs.} their respective densities. Indeed, while the sample complexity of OT and MMD stand at two extremes, $O(1/n^{1/d})$ for OT in dimension $d$ and $O(1/\sqrt{n})$ for MMD, that for SDs has only been studied empirically. In this paper, %we exhibit theoretical properties that shed light on the widely reported ability of Sinkhorn divergences to overfit less than OT. 
we \emph{(i)} derive a bound on the approximation error made with SDs when approximating OT as a function of the regularizer $\varepsilon$, \emph{(ii)} prove that the optimizers of regularized OT are bounded in a Sobolev (RKHS) ball independent of the two measures and \emph{(iii)} provide the first sample complexity bound for SDs, obtained,by reformulating SDs as a maximization problem in a RKHS. We thus obtain a scaling in $1/\sqrt{n}$ (as in MMD), with a constant that depends however on $\varepsilon$, making the bridge between OT and MMD complete.
\end{abstract}