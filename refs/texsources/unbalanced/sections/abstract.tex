% !TEX root = ../SinkhornDivergenceUnbalanced.tex



\abstract{
    Optimal transport (OT) distances (also called Wasserstein or Earth Mover's distances) are now routinely used to fit parametric models in data sciences.
    %
    They define geometric loss functions to compare point clouds or more generally probability distributions.
    %
    Their efficiency is however inhered by some lack of robustness to outliers, missing parts and sampling noise.
    %
    In this paper, we develop and analyze a new class of loss functions which combine two keys ideas to cope with these two robustness issues:
    (i) unbalanced optimal transport which relaxes the mass conservation constraint to lower sensitivity to outliers ;
    (ii) entropic regularization, which reduces the impact of sampling (especially in high dimension)  and lends itself to fast computations using the Sinkhorn algorithm.
    % induces the Earth Mover's (Wasserstein) distance between probability distributions, a geometric divergence that is relevant to a wide range of problems.
    % Over the last decade, two relaxations of optimal transport have been studied in depth: unbalanced transport, which  \cor{ambiguous}{robust} to the presence of outliers and can be used when distributions don't have the same total mass; entropy-regularized transport, which is \cor{ambiguous}{robust} to sampling noise and lends itself to fast computations using the Sinkhorn algorithm.
    %
    Our first set of contributions is the study of this new loss function, the so-called unbalanced Sinkhorn divergence, and we prove it is convex, positive, definite, and metrizes the convergence in law.
    %
    Our second set of contributions is the analysis of the associated Sinkhorn's algorithm, and we show its linear convergence for a wide set of unbalanced settings.
    %
    We provide numerical experiments for gradient flows and 3D scene flow estimation, showcasing the impact of this gain of robustness for applications to shape registration.
%    AMS :   68Q25, 68R10, 68U05
}
