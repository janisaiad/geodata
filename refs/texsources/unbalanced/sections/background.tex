\section{Background on Csiszár-divergences, Softmin and anisotropic proximity operators}
\label{sec-operators}

We present in this section concepts and properties required to study $\OTb$, $\Sb$, and the Sinkhorn algorithm.
We start with general properties of Csiszàr divergences $\D_\phi$, then focus on two operators called Softmin and anisotropic proximity operator involved in the analysis of Sinkhorn algorithm.

%Throughout this paper, we rely on structured functional operators to penalize deviations to the mass transportation constraints and perform or dampen the iterations of the Sinkhorn algorithm.
%Prior to our work on the optimal transport problem,
%we now introduce these objects rigorously and state their main properties.
%%
%Their link to our method will be detailed in Proposition~\ref{prop-optimality-prox}, which reformulates the Sinkhorn iterations in a convenient and numerically stable way.


%%%%%%%%%%%%%%%%%%%%%%%%%%%%%%%%%%%%%%%%%%%%%%%%%%%%%%%%%%
\subsection{Csiszár divergences}

We recall that entropy functions and Csiszàr divergences are defined in the introduction (see Equation~\ref{eq-csiszar-div}).
%
Some of their main properties are detailed below:

\begin{proposition}~\cite[Corollary (2.9)]{liero2015optimal}\label{prop-f-div-liero}
For any entropy function $\phi$, the divergence $(\al,\be)\mapsto\D_\phi(\al|\be)$ is positive, jointly convex, 1-homogeneous and weak* lower semicontinuous in $(\al,\be)$.
\end{proposition}

The \emph{Legendre conjugate} $\phi^*:\R\rightarrow\R$ of an entropy function $\phi$ is defined as $\phi^*(q) \eqdef \sup_{p\geq 0} pq - \phi(p)$.
%
The function $\phi^*$ appears in the dual formulation of $\OTb$, and has the following properties.


\begin{proposition}[Properties of the entropy conjugate $\phi^*$]\label{prop-legendre-conj}
For any entropy function $\phi$,
\begin{enumerate}
  \item One has $\partial{\phi^*} \subset\R_+$, i.e. $\phi^*$ is non-decreasing.
  \item The domain of $\phi^*$ is $(-\infty, \phi^\prime_\infty)$.
  \item One has $\lim_{q\rightarrow -\infty} \phi^*(q) = -\phi(0)$ and $\lim_{q\rightarrow +\infty} \phi^*(q) = +\infty$.
\end{enumerate}
\end{proposition}
\begin{proof}
%A property of Legendre transform in~\cite[Lemma 7.15]{santambrogio2015optimal} gives that $\partial\phi^*(q) = \arg\max\{ p\geq 0,\, \phi^*(q) = pq - \phi(p)\} \subset \text{dom}(\phi)\subset\R_+$. Thus $\partial\phi^* \subset\R_+$ and $\phi^*$ is non-decreasing.
Take $q\leq q'$. Because $\text{dom}(\phi)\subset\R_+$, for any $x\in\text{dom}(\phi)$ one has $xq - \phi(x) \leq xq' - \phi(x)$. Taking the supremum in $x$ gives $\phi^*(q) \leq \phi^*(q')$. Since $\phi^*$ is convex and non-decreasing we get $\partial{\phi^*} \subset\R_+$.

Assume $\phi^\prime_\infty < \infty$ and take $q > \phi^\prime_\infty$, $p>0$. Then one has $\lim_{p\rightarrow +\infty} p(q - \frac{\phi(p)}{p}) = +\infty$, i.e. $q\notin\text{dom}(\phi^*)$. If $\phi^\prime_\infty = \infty$ then for any $q\in\R$ $p\mapsto pq - \phi(p)$ goes to $-\infty$ when $p\rightarrow +\infty$, which gives coercivity in $p$ and guarantees that $\phi^*(q)$ is finite, i.e. $q\in\text{dom}(\phi^*)$.

By definition one has $\phi^*(q) \geq -\phi(0)$. When $q\rightarrow -\infty$, if $p>0$ then $pq - \phi(p)\rightarrow -\infty$. Thus we necessarily have $p=0$ and in that case it gives $\lim_{-\infty} \phi^* = -\phi(0)$. when $q\rightarrow +\infty$, because $\phi$ is an entropy function, we have that $\phi^*(q) \geq q.1 -\phi(1) = q$, which gives that $\lim_{+\infty} \phi^* = +\infty$.
\end{proof}

\begin{remark}\label{rem-param-rho}
For unbalanced OT, one can add a parameter $\rho >0$ so as to tune the strength of the mass conservation, and use $\D_{\rho\phi}=\rho\D_\phi$. 
Note that $(\rho\phi)^*(q) = \rho\phi^*(q / \rho)$.
One retrieves balanced OT when $\rho \rightarrow\infty$ (provided $\phi^{-1}(\{0 \})=\{ 1 \}$).
\end{remark}

%%%%%%%%%%%%%%%%%%%%%%%%%%%%%%%%%%%%%%%%%%%%%%%%%%%%%%%%%%
\subsection{Softmin operator}

The Softmin operator is a smoothed version of the minimum operator.


\begin{definition}[Softmin operator]\label{def-smin}
For any $\al \in \Mmpp(\Xx)$ and $\epsilon >0$, the Softmin operator $\Smin{\al}$ is such that for any $\f\in\Cc(\Xx)$
\begin{align}
\Smin{\al}(\f) \eqdef -  \epsilon \log\dotp{\al}{\exp(-\f/\epsilon)}.
\end{align}
\end{definition}

We detail some properties of this operator: these are helpful to get insights on its behaviour and are used extensively in subsequent proofs.

\begin{proposition}[Properties of the Softmin operator] \label{prop-smin-interp}
	For any $\epsilon>0$, Softmin is continuous w.r.t inputs $(\al,\f)$.
	It interpolates between a minimum operator and a sum, it is order preserving, and it is translation invariant.
	Those properties respectively read
	\begin{gather*}
	\big( \al_n \rightharpoonup\al \text{   and   } \f_n \xrightarrow{\norm{.}_\infty~}\f \big)
		\Longrightarrow\Smin{\al_n}(\f_n)\rightarrow\Smin{\al}(\f),\\
		\forall \al\in\Mmpo(\Xx),\, \dotp{\al}{\f}\xleftarrow{\epsilon\rightarrow +\infty} \Smin{\al}(\f)
		\xrightarrow{\epsilon\rightarrow 0}\min_{ x \in \Supp(\al)} \f(x), \\
		\forall(\f,\g)\in\Cc(\Xx), \, \f \leqslant\g \Longrightarrow\Smin{\al}(\f) \leqslant~\Smin{\al}(\g), \label{eq:smin_order} \\
		\forall K\in\R, \, \Smin{\al}(\f + K) = \Smin{\al}(\f) + K. \label{eq:smin_constant}
	\end{gather*}
\end{proposition}



We now mention some regularity properties of the Softmin.


\begin{lemma}[The Softmin operator is non-expansive]\label{lem-smin-lipschitz-func}
For any $\al\in\Mmpp(\Xx)$, the Softmin is $1$-Lipschitz. 
It is a non-expansive operator, with
\begin{align*}
 \forall (\f,\g)\in\Cc(\Xx), \quad
 |\Smin{\al}(\f) - \Smin{\al}(\g)| &\leq \norm{\f - \g}_\infty.
\end{align*}
\end{lemma}
\begin{proof}
Write $u_t = t(\g - \f) + \f$ for $t\in [0,1]$. 
The function $u_t$ is $\al$-measurable on a compact set, thus the function $t\mapsto \Smin{\al}(u_t)$ is differentiable. 
It gives
\begin{align*}
  |\Smin{\al}(\g) - \Smin{\al}(\f)| &= |\int_0^1 \frac{\d}{\d t} \Smin{\al}(u_t)| 
  = | \int_0^1 \dotp{\al}{(\g - \f)\frac{e^{u_t / \epsilon} }{ \dotp{\al}{ e^{u_t / \epsilon} } }} | \\
  &\leq  \int_0^1 |\dotp{\al}{(\g - \f)\frac{e^{u_t / \epsilon} }{ \dotp{\al}{ e^{u_t / \epsilon} } }} | 
  \leq \norm{\g - \f}_\infty.
\end{align*}
\end{proof}


We define two maps $\Ss_\al:\Cc(\Xx)\rightarrow\Cc(\Xx)$ and $\Ss_\be:\Cc(\Xx)\rightarrow\Cc(\Xx)$ derived from the Softmin. 
For any $(\f,\g)\in\Cc(\Xx)^2$ and $(x,y)\in\Xx^2$, the outputs $(\Ss_\al(\f), \Ss_\be(\g))$ read
%
\begin{align}\label{eq-defn-softmin-func}
	\Ss_\al(\f)(y)\eqdef \Smin{\al}(\C(\cdot,y) - \f),\qandq
	\Ss_\be(\g)(x)\eqdef \Smin{\be}(\C(x, \cdot) - \g).
\end{align}
Those maps are at the heart of Sinkhorn algorithm which solves the dual of~\eqref{eq-primal-unb}. We present the properties of $\Ss_\al(\f)$ (which hold analogously for $\Ss_\be(\g)$).

\begin{lemma}[Regularity of $\Ss_\al(\f)$]\label{lem-smin-cost-regular}
	Assume $\C$ is continuous on $\Xx^2$. For any $\al$-integrable function $\f$, $\Ss_\al(\f)$ is a continuous function.
	%
	If $\C$ is $\gamma$-Lipschitz in each of its inputs, then $\Ss_\al(\f)$ is $\gamma$-Lipschitz.
\end{lemma}
\begin{proof}
	The function $\f$ is $\al$-integrable and $\C$ is continuous on $\Xx$ compact, thus $x\mapsto\C(.,x)$ is uniformly bounded w.r.t. $x$.
	The dominated convergence theorem holds and $x\mapsto\dotp{\al}{e^{\frac{\f(.) - \C(.,x)}{\epsilon}}}$ is continuous.
	Concerning the Lipschitz property, Lemma~\ref{lem-smin-lipschitz-func} gives
	\begin{align*}
	|\Smin{\al}(\C(x,.) - \f) - \Smin{\al}(\C(y,.) - \f) | &\leq \norm{\C(x,.) - \C(y,.)}_\infty \\
	&\leq \gamma\d_\Xx(x,y).
	\end{align*}
\end{proof}




%%%%%%%%%%%%%%%%%%%%%%%%%%%%%%%%%%%%%%%%%%%%%%%%%%%%%%%%%%%%%%%
\subsection{Anisotropic proximity operator}

The maps $(\Ss_\al,\Ss_\be)$ suffice to define the \emph{balanced} Sinkhorn algorithm. 
The unbalanced version also involves the \emph{anisotropic proximity operator}, introduced in~\cite{combettes2013moreau,teboulle1992entropic}. 
It generalizes the usual \emph{proximal} operator from Hilbert spaces to Banach spaces. We start with its definition.

%One of the key remarks of this paper is detailed
%in Proposition~\ref{prop-optimality-prox}:
%in the unbalanced setting, we can define generalized
%Sinkhorn iterations by interleaving the ``standard''
%updates of the dual potentials with the pointwise
%application of an \emph{anisotropic proximity operator}.
%This object is studied in~\cite{combettes2013moreau, teboulle1992entropic} and generalizes the usual \emph{proximal} operator from Hilbert spaces to Banach spaces.

% Definition prox
\begin{definition}[Aprox operator]\label{def-prox}
Let $h : \R \rightarrow \R$ be a convex function and $\epsilon > 0$. The anisotropic proximity operator is defined as
\begin{align}\label{eq-def-aprox}
  \forall p \in \R, \quad
  \aprox{h}(p) \eqdef \arg\min_{q\in\R} \epsilon \exp(\tfrac{p - q}{\epsilon}) + h(q)\in \text{dom}(h)\cup\{+\infty\}.
\end{align}
%This operator is noted $\aprox_{f}$ when $\epsilon = 1$. When $f=\phi^*$, one has the relation
%\begin{align}\label{eq-prox-eps-rel}
%  \aprox_{\phi^*}^{\epsilon}(p) = \epsilon\aprox_{(\phi/\epsilon)^*}(p/\epsilon).
%\end{align}
\end{definition}

If there exists $x\in\text{dom}(h)$ such that $\partial h(x)\subset\R_+^*$, then for any $p\in\R$, $\aprox{h}(p) < +\infty$.
It holds when $h=\phi^*$, see Proposition~\ref{prop-legendre-conj}.

% When $h=\phi^*$,  $\aprox{\phi^*}$ is well-defined (Proposition~\ref{prop-optimality-prox}).
 As detailed in \cite{combettes2013moreau}, a generalized Moreau decomposition connects it with a $\KL$ (Bregman) proximity operator that reads
\begin{align*}%\label{eq-aprox-proxkl}
  \aprox{\phi^*}(p) &=  p -  \epsilon\log \proxdiv{\phi}(p),\\
  	\quad\text{where}\quad \proxdiv{\phi}(p)&\eqdef \arg\inf_{q\in\R_+} \phi(q) + \KL(q, \exp(\tfrac{p}{\epsilon}))
\end{align*}
%
The above $\proxdiv{\phi}$ operator is used in~\cite{chizat2016scaling} to define the Sinkhorn algorithm.
We present below one advantage of $\aprox{\phi^*}$, namely its non-expansiveness, similarly to the standard proximal operators.
It is key in Section~\ref{sec-sinkhorn} to prove convergence of Sinkhorn algorithm in wide generality.
%
%The following proposition shows that the anisotropic proximity operator is nonexpansive in its input. This property is used in Section~\ref{sec-sinkhorn} to prove that the Sinkhorn algorithm converges in wide generality.
%
%(Note that $\KL$ is the only divergence that is both a Bregman and a $\phi$-divergence.)

\begin{proposition}[The aprox is non-expansive]\label{prop-nonexp}
For any entropy $\phi$, the anisotropic proximity operator is $1-$Lipschitz. 
For any $(p,q)\in\R$, one has
\begin{align*}\label{eq-aprox-nonexp}
  \norm{\aprox{\phi^*}(p) - \aprox{\phi^*}(q)}_\infty \leq |p - q|.
\end{align*}
\end{proposition}
%
The proof relies on properties of monotone operators, and is deferred to Appendix~\ref{appendix-proofs}.
%
%\begin{proof}
% Take two pairs $(p_1,q_1)$, $(p_2,q_2)$ such that for $i\in\{1,2\}$, $q_i = \aprox_{\phi^*}(p_i)$. This is equivalent to $e^{p_i-q_i}\in\partial\phi^*(q_i)$, and because $\partial\phi^*$ is a monotone operator one has
% \begin{align*}
%   (e^{p_1-q_1} - e^{p_2-q_2})(q_1 - q_2) \geq 0.
% \end{align*}
% Then one can use the first order convexity condition to get
% \begin{align*}
%   &e^{p_1-q_1} - e^{p_2-q_2} \geq e^{p_2-q_2}(p_1 - q_1 - p_2 + q_2),\\
%   &e^{p_2-q_2} - e^{p_1-q_1} \geq e^{p_1-q_1}(p_2 - q_2 - p_1 + q_1),\\
%   &\Rightarrow 0\geq (e^{p_1-q_1} - e^{p_2-q_2})(p_2 - q_2 - p_1 + q_1)\\
%   &\Rightarrow (e^{p_1-q_1} - e^{p_2-q_2})(p_1 - p_2) \geq (e^{p_1-q_1} - e^{p_2-q_2})(q_1 - q_2)\geq 0.
% \end{align*}
% The case $e^{p_1-q_1} = e^{p_2-q_2}$ is trivial, and without loss of generality we can assume $e^{p_1-q_1} - e^{p_2-q_2} > 0$ by swapping indices if necessary. Eventually it gives the pointwise inequality
% \begin{align*}
%   |p_1-p_2| \geq |q_1 - q_2| = |\aprox_{\phi^*}(p_1) - \aprox_{\phi^*}(p_2)|.
% \end{align*}
% The above inequality gives that if $x\mapsto p(x)$ is a continuous function instead of a real number, then $x\mapsto\aprox_{\phi^*}(p(x))$ is also a continuous function (Take $p_1 = p(x)$, $p_2 = p(y)$ and let $x\rightarrow y$). Now take $q_1 = \aprox_{\phi^*}(\f)$ and $q_2 = \aprox_{\phi^*}(\g)$ for some $(\f,\g)\in\Cc(\Xx)$. Since $\Xx$ is compact, suprema are attained and we can take the point $x\in\Xx$ such that
% \begin{align*}
%   \norm{q_1-q_2}_\infty = |q_1(x) - q_2(x)|\leq |\f(x) - \g(x)|\leq \norm{\f - \g}_\infty.
% \end{align*}
% This proves the statement for $\epsilon=1$, and Equation~\eqref{eq-prox-eps-rel} allows to conclude for any $\epsilon>0$.
%\end{proof}
%
We end with a monotonicity property on the aprox.

\begin{proposition}[The aprox is non-decreasing]\label{prop-monoton-aprox}
  For any entropy $\phi$ with Legendre transform $\phi^*$, the operator $\aprox{\phi^*}$ is non-decreasing.
\end{proposition}

\begin{proof}
  Assume $\phi^*$ is smooth, and write $g(p) \eqdef \aprox{\phi^*}(p)$. 
  The implicit function theorem holds and yields differentiability of aprox. 
  Its derivative reads
  $\phi^{*\prime}(g(p)) = e^{\frac{p - g(p)}{\epsilon}}$,
  which implies that $g^\prime(p) = \frac{\phi^{*\prime}(g(p))}{\phi^{*\prime}(g(p)) + \phi^{*\prime\prime}(g(p))} \in[0,1],$
%	\begin{align*}
%		g^\prime(p) = \frac{\phi^{*\prime}(g(p))}{\phi^{*\prime}(g(p)) + \phi^{*\prime\prime}(g(p))} \in[0,1],
%	\end{align*}
since $\phi^*$ is convex and non-decreasing (Proposition~\ref{prop-legendre-conj}), we have $\phi^{*\prime},\phi^{*\prime\prime}\geq 0$.

If $\phi^*$ is not smooth then one can regularize it, and let the regularization go to zero. It yields a sequence of operators $(\aprox{\phi_n^*})_n$ which are non-decreasing, and that converges pointwise to $\aprox{\phi^*}$. Due to closedness of non-decreasing functions, the limit is also non-decreasing, hence the result.
\end{proof}


%%%%%%%%%%%%%%%%%%%%%%%%%%%%%%%%%%%%%%%%%%%%%%%%%%%%%%%%%%%%%%


