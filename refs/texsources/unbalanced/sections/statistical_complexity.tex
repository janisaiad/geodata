\section{Statistical Complexity of Unbalanced Transport}
\label{sec-stat-comp}

A common assumption in statistics, machine learning and imaging is that one does not have directly access to the distributions $(\al,\be)$, but rather that the data is composed of a set of $n$ samples from these measures. 
%
Thus an important theoretical and practical question is the study of the discretization error made when approximating $\OTb(\al,\be)$ with $\OTb(\al_n,\be_n)$. 

More precisely, we wish to establish the convergence rate of $|\OTb(\al_n,\be_n)- \OTb(\al,\be)|$ as $n\rightarrow\infty$ so as to know how many samples are needed to reach a desired tolerance error. 
For unregularized OT the rate is $O(n^{-1 / d})$ when $\Xx=\R^d$~\cite{dudley1969speed}. 
It was refined in~\cite{weed2017sharp} to be $O(n^{-1 / d^*})$ where $d^*$ is a quantification of the intrinsic dimension of the measure.
%
Entropic regularization has been proved to mitigate this curse of dimensionality, yielding in $\R^d$ when $\epsilon\rightarrow 0$ a rate of $O(\epsilon^{- \lfloor d / 2 \rfloor}n^{-1/2})$~\cite{genevay2018sample}, with an improvement of the dependency with $\epsilon$ of the constant in~\cite{mena2019statistical} (which also extends this result from compact domains to sub-Gaussian measures).

A recent work shows the statistical and time benefits of using $\Sb$ in the Balanced case instead of $\OTb$~\cite{chizat2020faster}.
It allows to obtain accurate approximations of $\OT$ while allowing a larger regularization $\epsilon$ compared to using $\OTb$.
%
%Under a mild regularity assumption on the $(\f,\g)$, they prove that $|\Sb(\al_n,\be_n) - \OT(\al,\be)|=O(n^{-2/d})$ when $d>4$. \fx{A priori ça dépend de epsilon non?}
%
This proof relies on a dynamic formulation of entropic OT.
An unbalanced entropic dynamic formulation was recently developed in~\cite{baradat2021regularized}, but its only connected to $\OTb$ when $\D_\phi=\TV$ and $\C(x,y)=|x-y|^2$.
In this section we consider general (but smooth) $\phi$ and $\C$, which excludes the TV case.
For this reason, our results which focuses on $\OTb$ instead of $\Sb$ remain of interest to the community.


This section extends the results of~\cite{genevay2018sample,mena2019statistical} to the framework of unbalanced OT. 
%
We suppose in addition with all the previous assumptions that the cost $\C$ and the function $\phi^*$ are $\Cc^\infty$. We assume the space $\Xx$ is a compact Lipschitz domain of $\R^d$. 

We denote by $(\al,\be)\in\Mmp(\Xx)$ the input positive measures, by $(\bar{\al},\bar{\be})\in\Mmpo(\Xx)$ their normalized versions and by $(\al_n,\be_n)$ their empirical counterparts with $n$ points, i.e.
\begin{align*}
  \al = m(\al)\bar{\al}, \quad \al_n = \frac{m(\al)}{n} \sum_{i=1}^n \de_{X_i} \qquad \be = m(\be)\bar{\be}, \quad \be_n = \frac{m(\be)}{n} \sum_{i=1}^n \de_{Y_i},
\end{align*}
where $(X_1,...,X_n)$ and $(Y_1,...,Y_n)$ are $n$ points in $\Xx$ sampled from the normalized probability distributions $(\bar{\al},\bar{\be})$. 
Note that we assume for simplicity that the masses of $(\al,\be)$ are known, so that the total masses of $(\al_n,\be_n)$ are the same as those of $(\al,\be)$.

The main result of this section is the following theorem. Its proof is very technical and is detailed in Appendix~\ref{appendix-statistical-complexity}.
%
\begin{theorem}[Sample complexity of the unbalanced transport cost]\label{thm-sample-complexity-unb}
Assume $\phi^*$ and  $\C$ are $\Cc^\infty$ and that Assumptions~(\ref{as:1}, \ref{as:2}) hold. 
Then there exists a rational fraction $\Qq(\epsilon)$ whose coefficients only depend on the norms $\norms{\C^{(k)}}_\infty$ and $\norms{\phi^{*(k)}}$, respectively evaluated on compact sets $\Xx$ and $\Yy$ where $\Yy$ is a compact independent of $\epsilon$, such that
\begin{align*}
  For\, any \,\, \epsilon,\;\; \mathbb{E}_{\bar{\al}\otimes\bar{\be}}\big[|\OTb(\al,\be) - \OTb(\al_n,\be_n)|\big] = O\bigg(\frac{m(\al) + m(\be)}{\sqrt{n}}\Qq(\epsilon)\bigg).
\end{align*}
%
Furthermore the rational fraction has the following asymptotics.
%\begin{align*}
%  For\,\, \epsilon\rightarrow 0,\quad \mathbb{E}_{\bar{\al}\otimes\bar{\be}}\big[|\OTb(\al,\be) - \OTb(\al_n,\be_n)|\big] = O\bigg(\frac{m(\al) + m(\be)}{\epsilon^{\lfloor d/2 \rfloor} \sqrt{n}}\bigg),\\[0.5em]
%  For\,\, \epsilon\rightarrow \infty,\quad \mathbb{E}_{\bar{\al}\otimes\bar{\be}}\big[|\OTb(\al,\be) - \OTb(\al_n,\be_n)|\big] = O\bigg(\frac{m(\al) + m(\be)}{\sqrt{n}}\bigg).
%\end{align*}
\begin{align*}
\Qq(\epsilon) = O_{\epsilon\rightarrow 0}(\epsilon^{-\lfloor d/2 \rfloor}) \qandq \Qq(\epsilon) = O_{\epsilon\rightarrow\infty}(1).
\end{align*}
\end{theorem}

The proof of this result relies on several lemmas presented in Appendix~\ref{appendix-statistical-complexity}. 
In particular, we show dual potentials are smooth and belong to a Sobolev space $\Hh^s_\al(\Xx)$, which is a RKHS when $s > \lfloor \tfrac{d}{2} \rfloor$. %
Note that for a given dimension $d$, it suffices to assume $\C$ and $\phi^*$ are $\Cc^{\lfloor d /2 \rfloor + 1}$. 
We show that the potentials lie in a ball of $\Hh^s_\al(\Xx)$ endowed with its corresponding Sobolev norm, 
Then we apply standard results from the PAC-learning theory in Reproducing Kernel Hilbert Spaces.
%
\begin{proof}
We first start by applying Proposition~\ref{prop-ineq-ot-sob}
\begin{align*}
 \mathbb{E}_{\bar{\al}\otimes\bar{\be}}\big[|\OTb(\al,\be) - \OTb(\al_n,\be_n)|\big]  &\leq 2 \mathbb{E}_{\bar{\al}}\big[\sup_{\f\in\Hh^s_{\al,\lambda}(\R^d)} |\dotp{\al - \al_n}{\f}|\big] \\
  &+2 \mathbb{E}_{\bar{\be}}\big[\sup_{\f\in\Hh^s_{\be,\lambda}(\R^d)} |\dotp{\be - \be_n}{\g}|\big].
\end{align*}

% The application $\g\mapsto\g$ is $1$-Lipschitz. 
Write $\al = m(\al) \bar{\al}$ and $\be = m(\be) \bar{\be}$. We apply Proposition~\ref{prop-pac-rkhs} with $B=1$ to the Sobolev space $\Hh^s_{\al,\lambda}(\R^d)$ with $s = \lfloor \tfrac{d}{2}\rfloor + 1$, such that $\Hh^s_{\al,\lambda}(\R^d)$ and $\Hh^s_{\be,\lambda}(\R^d)$ are RKHS. It yields for the normalized measure $\bar{\al}\in\Mmpo(\Xx)$
\begin{align*}
  \mathbb{E}_{\bar{\al}}\bigg[ \sup_{\f\in\Hh^s_{\al,\lambda}(\R^d)} |\dotp{\bar{\al} - \bar{\al}_n}{\f}| \bigg] \leq \frac{2\lambda}{\sqrt{n}}
%  \Rightarrow \mathbb{E}_{\bar{\al}}\bigg[ \sup_{\f\in\Hh^s_{\al,\lambda}(\R^d)} |\dotp{\al - \al_n}{\f}| \bigg] \leq m(\al)\frac{2\lambda}{\sqrt{n}}
\end{align*}
where $\lambda=\Qq(\epsilon)$ is the radius of the Sobolev ball bounding the potentials (Proposition~\ref{prop-dual-pot-sob}). 
We get the desired result by multiplying by $m(\al)$, and summing with the similar term obtained fo $\be$.
\end{proof}

