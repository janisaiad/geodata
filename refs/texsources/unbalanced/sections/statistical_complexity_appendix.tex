\section{Sample Complexity - Proof of Theorem~\ref{thm-sample-complexity-unb}}
\label{appendix-statistical-complexity}
\subsection{Prerequisites}

We present in this section the material which is necessary to follow the details of the proof.
%
We first define Sobolev spaces, and then detail the Faà Di Bruno which is extensively applied in the proofs, with the main result on sample complexity in RKHS.

\begin{definition}
The Sobolev space $\Hh^s_\al(\Xx)$, for $s\in\N^*$, is the space of functions $\f:\Xx\rightarrow \R$ such that for every multi-index $k$ with $|k|\leq s$, the mixed partial derivative $\f^{(k)}$ exists and belongs to $\mathbb{L}^2_\al(\Xx)$. It is endowed with the inner-product
\begin{align*}%\label{def-sobolev-prod}
  \dotp{\f}{\g}_{\Hh^s_\al(\Xx)} \eqdef \sum_{|k|\leq s} \int_\Xx \f^{(k)}(x)\g^{(k)}(x) \d\al.
\end{align*}

We also define the Sobolev ball $\Hh^s_{\al,\lambda}(\Xx) = \{ \f\in\Hh^s_\al(\Xx), \, \norms{\f}_{\Hh^s_\al(\Xx)} \leq \lambda\}$.
\end{definition}

We recall that for $s > \lfloor \tfrac{d}{2} \rfloor$, $\Hh^s(\R^d)$ is a RKHS. Furthermore the Sobolev extension theorem~\cite{calderon1961lebesgue} gives that $\norms{.}_{\Hh^s_\al(\R^d)} \leq C \norms{.}_{\Hh^s_\al(\Xx)}$ provided that $\Xx$ is a bounded Lipschitz domain. Thus, in what follows it suffices to control the dual potentials with respect to the norm $\norms{.}_{\Hh^s_\al(\Xx)}$ over the compact $\Xx$.

We now state the PAC-learning result we apply in $\Hh^s(\R^d)$. It is a combination of the proofs of Theorem (8), (12.4) and Lemma 22 in~\cite{bartlett2002rademacher}.
\begin{proposition}\label{prop-pac-rkhs}
\cite{bartlett2002rademacher} 
Consider $\al\in\Mmpo(\Xx)$, a $B$-Lipschitz loss $L$ and $\Gg$ a given class of functions. Then
\begin{align*}%\label{ineq-pac-rkhs}
  \mathbb{E}_\al \big[ \sup_{\f\in\Gg} \mathbb{E}_\al L(\f) - \mathbb{E}_{\al_n} L(\f) \big] \leq 2B \mathbb{E}_\al\Rr(\Gg)
\end{align*}
where $\mathbb{E}_\al[\f]=\dotp{\al}{\f}$ and $\Rr(\Gg)$ denotes the Rademacher complexity of the class of functions $\Gg$. When $\Gg$ is a ball of radius $\lambda$ in a RKHS with kernel $k$ the Rademacher complexity is bounded by
\begin{align*}
  \mathbb{E}_\al\Rr(\Gg) \leq \frac{\lambda}{n}\sqrt{\sum_{i=1}^n k(X_i,X_i)} \leq \frac{\lambda}{\sqrt{n}} \sqrt{\max_{x\in\Xx} k(x,x)}.
\end{align*}
\end{proposition}

The loss defined in this property will be the identity which is 1-Lipschitz, while the function used in~\cite{genevay2018sample} had a Lipschitz constant depending exponentially in $\epsilon$. 

In order to prove that the potentials are in such RKHS, we need to explicit the derivatives of the potentials through a differentiation of the Sinkhorn mapping. Since it is a composition of several functions, we need to use the Faà Di Bruno formula. It has been generalised for the composition of multivariate functions in~\cite{constantine1996multivariate}. We detail a corollary of the general formula because we only need a composition of function where only the first one is multivariate.

\begin{proposition}\label{prop-faadibruno}
\cite[Corollary 2.10]{constantine1996multivariate}
Define the functions $\g:\R^d \rightarrow \R$, $\f:\R\rightarrow\R$ and $h=\f\circ\g$. Take $x\in\R^d$, $y=\g(x)\in\R$ and $n=|\nu|\in\mathbb{N}\setminus\{0\}$ where $\nu$ is a multi-index. Assume $\g$ is $\Cc^\nu$ at $x$ and $\f$ is $\Cc^n$ at $y$. Then
\begin{align*}
  h^{(\nu)}(x) = \sum_{\lambda=1}^n \f^{(\lambda)}(y) \sum_{p(\nu,\lambda)} (\nu!) \prod_{j=1}^n \frac{\big(g^{(l_j)}(x)\big)^{k_j}}{(k_j!)(l_j!)^{k_j}},
\end{align*}
where
\begin{align*}
  p(\nu,\lambda) = \{& (k_1,...,k_n)\in (\N)^n, \, (l_1,...,l_n)\in (\N^d)^n,\, \exists s\in\llbracket 1,n\rrbracket, \forall i \in\llbracket s,n\rrbracket,\,\\ &k_i>0 \text{ and } 0\prec l_s \prec ... \prec l_n \, \text{ such that } 
  \sum_{i=1}^n k_i = \lambda,\, \sum_{i=1}^n k_i l_i = \nu\}.
\end{align*}

The $0$-th derivative is the function itself. The factorial of a vector is the product of the factorial of the coordinates. One has $l\prec \tilde{l}$ when either $|l| < |\tilde{l}|$ or when $|l| = |\tilde{l}|$ it is larger w.r.t. the lexicographic order. In the monovariate setting, we necessarily have $l_s = s$.
\end{proposition}

\subsection{Proof of the sample complexity}

Terms of the form $\phi^{*(k)}(-\f)$ will appear in the derivation of the bounds. 
%
Since we are looking at the dependence in $\epsilon$, and because the optimal potential $\f$ implicitly depends on it, we need this first lemma which asserts that its norm is uniformly bounded independently of $\epsilon$.
%
Knowing that the dual potentials do not diverge with respect to $\epsilon$ allows to consider a compact $\Yy$ in which $\norms{\phi^{*(k)}}_\infty$ is finite.
%
In what follows, the norms $\norms{\C^{(k)}}_\infty$ and $\norms{\phi^{*(k)}}_\infty$ are meant to be estimated on $\Xx$ and $\Yy$ respectively. Since $\C$ and $\phi^*$ are $\Cc^\infty$, those norms are all finite.

\begin{proposition}\label{prop-unif-bound-pot-eps}
Take any pair of measures $(\al,\be)\in\Mmpp(\Xx)$. Under Assumption~(\ref{as:1},\ref{as:2}), the potentials are uniformly bounded by a bound which is independent of $\epsilon$.
\end{proposition}
\begin{proof}
Lemma~\ref{lem-uniq-tensor-sum} holds and asserts that the dual functional is strictly convex and coercive in $\f\oplus\g$. Though, coercivity when $\f\oplus\g \rightarrow + \infty$ seems to depend on $\epsilon$ because of the term $\epsilon(\tefgc - 1)$. Since $\epsilon(e^{x / \epsilon}-1) \geq x$ for any $x$, coercivity is guaranteed independently of $\epsilon$, and one gets that $\norms{\f\oplus\g}_\infty$ is uniformly bounded independently of $\epsilon$.
%
It remains to prove the same property for $\f$ and $\g$. Any optimal potential $\f$ is $\gamma$-Lipschitz. thus if one writes $\f = \lambda + h$ with $h(x_0) = 0$, $h$ is also Lipschitz, thus $\norms{h}_\infty \leq \gamma\text{diam}(\Xx)$. It remains to prove that $\lambda$ can be uniformly bounded independently of $\epsilon$.
%
The proof of Lemma~\ref{lem-compact-dual} shows that under Assumptions~\ref{as:2}, the dual functional is coercive under translations, due to the terms involving $\phi^*$ which do not depend on $\epsilon$.
Thus coercivity holds independently of $\epsilon$. We have $\norms{\f}_\infty \leq |\lambda| + \gamma\text{diam}(\Xx)$ where $\lambda$ is in a compact set independent of $\epsilon$.
\end{proof}



Before stating the result on the regularity of the dual potentials, we prove a technical proposition that explicits the expression of derivatives of the $\aprox{\phi^*}$ operator.
%
We introduce a generic notation by expressing some terms implicitly as polynomials of the parameter $\epsilon$ of order $k$, written $P_{k}(\epsilon)$. In some calculations the same notation $P$ is used to represent different objects from one line to another.

\begin{proposition}\label{prop-deriv-aprox}
Assume that $\phi^*$ is $\Cc^\infty$. Then the operator $\aprox{\phi^*}$ is also $\Cc^\infty$, and its n-th derivative verifies for any $n$
\begin{align*}
  (\aprox{\phi^*})^{(n)}(x) = \frac{P_{n-1}(\epsilon)(x)}{(\phi^{*\prime} + \epsilon\phi^{*\prime\prime})^{2n-1}}
\end{align*}
where $P_{n-1}(\epsilon)$ represents a polynomial in $\epsilon$ of order $n-1$ whose coefficients are functions which only depend on the derivatives of $\phi^*$ up to the order $n$. The dependance of $P_{n-1}(\epsilon)$ in $x$ only appears through the derivatives of $\phi^*$.
\end{proposition}
\begin{proof}
For sake of conciseness we will write $p(x) = \aprox{\phi^*}(x)$ in this proof.
The regularity of $\aprox{\phi^*}$ is given by the optimality condition of its definition, i.e. $\phi^{*\prime} (p(x)) = e^{(x - p(x)) / \epsilon}$.
This expression is a $\Cc^\infty$ function in $(x,p(x))$ whose derivatives are never nonzero, thus the implicit function theorem gives that $\aprox{\phi^*}$ is $\Cc^\infty$.

We prove the bound on the derivatives of $\aprox{\phi^*}$ by a strong induction. Differentiating this equation yields
\begin{align}
  p^\prime\phi^{*\prime\prime}\circ &p = \frac{1 - \phi^{*\prime}\circ p}{\epsilon} e^{\frac{x - p(x)}{\epsilon}}
  = \frac{1 - p^{\prime}}{\epsilon} \phi^{*\prime}\circ p\nonumber\\
&p^\prime(\phi^{*\prime}\circ p + \epsilon\phi^{*\prime\prime}\circ p) = \phi^{*\prime}\circ p. \label{eq-diff-optim-prox} 
\end{align}
This relation proves the statement for $n=1$. Let's assume now that the property is true up to a given integer $n$. Applying the Faà Di Bruno and Leibniz formulas~\ref{prop-faadibruno} to the above equation~\eqref{eq-diff-optim-prox} gives
\begin{align}
&p^{(n+1)} (\phi^{*\prime}\circ p + \epsilon\phi^{*\prime\prime}\circ p) = \label{eq-faaleibniz-line0}\\
  &\quad\sum_{\lambda=1}^n \phi^{*(\lambda+1)}\circ p \sum_{p(\nu,\lambda)} (\nu!) \prod_{j=1}^n \frac{\big(p^{(l_j)}\big)^{k_j}}{(k_j!)(l_j!)^{k_j}} \label{eq-faaleibniz-line1}\\
  &-\sum_{k=0}^{n-1}\binom{n}{k} p^{(k+1)}\sum_{\lambda=1}^{n-k} (\phi^{*(\lambda + 1)}\circ p + \epsilon\phi^{*(\lambda + 2)}\circ p) \sum_{p(\nu,\lambda)} (\nu!) \prod_{j=1}^n \frac{\big(p^{(l_j)}\big)^{k_j}}{(k_j!)(l_j!)^{k_j}}.\label{eq-faaleibniz-line2}
\end{align}

Note that in the above formula the last derivative~\eqref{eq-faaleibniz-line0} in the leibniz formula has been separated from the rest of the sum~\eqref{eq-faaleibniz-line2}. Applying the induction hypothesis, one gets that for any $\lambda$ line~\ref{eq-faaleibniz-line1} is a polynomial of order
\begin{align*}
  \prod_{j=1}^n \big(p^{(l_j)}\big)^{k_j} 
  &= \prod_{j=1}^n \frac{P_{k_j(l_j -1)}(\epsilon)}{(\phi^{*\prime} + \epsilon\phi^{*\prime\prime})^{k_j (2l_j - 1)}} \\
  &= \frac{P_{n - \lambda}(\epsilon)}{(\phi^{*\prime} + \epsilon\phi^{*\prime\prime})^{2n - \lambda}}  \times \frac{(\phi^{*\prime} + \epsilon\phi^{*\prime\prime})^{\lambda}}{(\phi^{*\prime} + \epsilon\phi^{*\prime\prime})^{\lambda}}
  = \frac{P_{n}(\epsilon)}{(\phi^{*\prime} + \epsilon\phi^{*\prime\prime})^{2n}}.
\end{align*}
As the same term appears line~\ref{eq-faaleibniz-line2} with a Faà Di Bruno formula that stops at the order $n-k$, one gets for any $(k,\lambda)$ a term of order
\begin{align*}
  \frac{P_{k}(\epsilon)}{(\phi^{*\prime} + \epsilon\phi^{*\prime\prime})^{2k+1}} &\times P_{1}(\epsilon) \times \frac{P_{n-k-\lambda}(\epsilon)}{(\phi^{*\prime} + \epsilon\phi^{*\prime\prime})^{2(n-k) - \lambda}}\\
&= \frac{P_{n+1 - \lambda}(\epsilon)}{(\phi^{*\prime} + \epsilon\phi^{*\prime\prime})^{2n - \lambda + 1}} \times \frac{(\phi^{*\prime} + \epsilon\phi^{*\prime\prime})^{\lambda - 1}}{(\phi^{*\prime} + \epsilon\phi^{*\prime\prime})^{\lambda - 1}} 
= \frac{P_n(\epsilon)}{(\phi^{*\prime} + \epsilon\phi^{*\prime\prime})^{2n}}.
\end{align*}
Eventually, dividing $p^{(n+1)} (\phi^{*\prime} + \epsilon\phi^{*\prime})$ by $(\phi^{*\prime} + \epsilon\phi^{*\prime\prime})$ gives the right denominator and ends the proof by strong induction.
\end{proof}



\begin{proposition} \label{prop-dual-pot-sob}
Assume that $\phi^*$ and  $\C$ are $\Cc^\infty$ and that Assumptions~(\ref{as:1}, \ref{as:2}) hold.
One has $\phi^*(-\f) + \epsilon\nabla\phi^*(-\f)\in\Hh^s_{\al,\lambda}(\R^d)$ and $\phi^*(-\g) + \epsilon\nabla\phi^*(-\g)\in\Hh^s_{\be,\lambda}(\R^d)$,
where the radius of the ball $\lambda$ is a rational fraction of $\epsilon$ with coefficients depending on the norms $\norm{\phi^{*(k)} }$ and $\norms{\C^{(k)} }$ for derivatives $k$ up to the order $s$, but is independent of the measures' masses. Its asymptotics for $\epsilon$ going to either $0$ or $+\infty$ read
\begin{align*}
  For \,\,\epsilon\rightarrow 0, \quad \lambda = O(1/ \epsilon^{s-1}), \qandq
  for \,\,\epsilon\rightarrow \infty, \quad \lambda = O(1).
\end{align*}
\end{proposition}
\begin{proof}
This proof applies several times the Faà Di Bruno formula~\ref{prop-faadibruno} to the function
\begin{align*}
  x\mapsto (\phi^* + \epsilon\nabla\phi^*)\circ\aprox{\phi^*}\circ(\epsilon\log)\circ\dotp{\be}{e^{\frac{\g-\C(x,.)}{\epsilon}}}.
\end{align*}

\hfill\break
\textbf{Differentiation under the integral.}
%
We differentiate the operator $x\mapsto\dotp{\be}{e^{(\g-\C(x,.)) / \epsilon}}$. An application of the dominated convergence theorem similar to Lemma~\ref{lem-smin-cost-regular} proves that it is as smooth as the cost $\C$ and that the differentiation and integration can be swapped. In other words
\begin{align*}
  \partial^{(k)}\dotp{\be}{e^{(\g-\C(x,.)) / \epsilon}} = \dotp{\be}{\partial^{(k)} e^{(\g-\C(x,.)) / \epsilon}}.
\end{align*}
Applying Proposition~\ref{prop-faadibruno} to $h = \exp\circ(-\C / \epsilon)$ defined on $\R^d\rightarrow\R\rightarrow\R$ gives
\begin{align*}
  h^{(\nu)}(x) &= \sum_{\lambda=1}^n e^{-\C(x,.) / \epsilon} \sum_{p(\nu,\lambda)} (\nu!) \prod_{j=1}^n \frac{\big(-\C^{(l_j)}(x,.) / \epsilon\big)^{k_j}}{(k_j!)(l_j!)^{k_j}}\\
  &= e^{-\C(x,.) / \epsilon} \sum_{\lambda=1}^n (\tfrac{1}{\epsilon})^\lambda \sum_{p(\nu,\lambda)} (\nu!) \prod_{j=1}^n \frac{\big(-\C^{(l_j)}(x,.)\big)^{k_j}}{(k_j!)(l_j!)^{k_j}}\\
  &\leq e^{-\C(x,.) / \epsilon} \sum_{\lambda=1}^n (\tfrac{1}{\epsilon})^\lambda \sum_{p(\nu,\lambda)} (\nu!) \prod_{j=1}^n \frac{\big(\norms{\C^{(l_j)}}_\infty\big)^{k_j}}{(k_j!)(l_j!)^{k_j}}
\end{align*}
Note that the norm $\norms{.}_\infty$ verifies $\norms{\f\g}_\infty\leq\norms{\f}_\infty\norms{\g}_\infty$.

Thus one can bound the derivative of the integral
\begin{align}\label{eq-bound-faa-int}
  \partial^{(k)}\dotp{\be}{e^{(\g-\C(x,.)) / \epsilon}} \leq \dotp{\be}{e^{(\g-\C(x,.)) / \epsilon}} Q_k(\tfrac{1}{\epsilon}),
\end{align}
where $Q_k$ is a polynomial in $1/\epsilon$ of order $k$ with no constant term (it is important when $\epsilon\rightarrow\infty$), whose coefficients only depend on the norm of the derivatives of $\C$.

\hfill\break
\textbf{Differentiation of the Sinkhorn mapping.}
We now differentiate the composition of $T(x) = -\aprox{\phi^*}(\epsilon\log(x))$ for any smooth $\aprox{\phi^*}$ operator. Given that $\log^{(\nu)}(x) = (-1)^\nu (\nu - 1)! x^{-\nu}$, the Faà Di Bruno formula~\ref{prop-faadibruno} with Proposition~\ref{prop-deriv-aprox} formula gives
\begin{align}
  T^{(\nu)}(x) &= -\sum_{\lambda=1}^n (\aprox{\phi^*})^{(\lambda)}(\epsilon\log(x)) \sum_{p(\nu,\lambda)} (\nu!) \prod_{j=1}^n \frac{\big( \epsilon(-1)^j (j - 1)! x^{-j} \big)^{k_j}}{(k_j!)(j!)^{k_j}} \label{eq-faa-sink-1} \\ 
  &= -(-1)^\nu x^{-\nu} \sum_{\lambda=1}^n (\epsilon)^\lambda(\aprox{\phi^*})^{(\lambda)}(\epsilon\log(x)) \sum_{p(\nu,\lambda)} (\nu!)  \prod_{j=1}^n \frac{1}{(k_j!)(j)^{k_j}}\label{eq-faa-sink-2}\\
  &= -(-1)^\nu x^{-\nu} \sum_{\lambda=1}^n \frac{(\epsilon)^\lambda P_{\lambda-1}(\epsilon)(x)}{(\phi^{*\prime} + \epsilon\phi^{*\prime\prime})^{2\lambda - 1}}\label{eq-faa-sink-3}\\
  &\leq x^{-\nu} \sum_{\lambda=1}^n \frac{(\epsilon)^\lambda P_{\lambda-1}(\epsilon)}{(\inf\phi^{*\prime} + \epsilon\inf\phi^{*\prime\prime})^{2\lambda - 1}}.\label{eq-bound-faa-sink}
\end{align}
We recall that the Faà Di Bruno formula imposes $\sum k_j = \lambda$ and $\sum j k_j=\nu$, hence the simplification from line~\eqref{eq-faa-sink-1} to line~\eqref{eq-faa-sink-2}. Line~\eqref{eq-faa-sink-3} is an application of Proposition~\ref{prop-deriv-aprox} which simplifies the expression. Eventually we can bound this term as displayed line~\eqref{eq-bound-faa-sink}. In this last line the polynomial is meant to depend on the norms $\norms{\phi^{*(k)}}$ and $\epsilon$ but not on $x$ (since $x$ appears through the derivatives of $\phi^*$).


\hfill\break
\textbf{Differentiation of the dual potential.}
%
Applying the Faà di Bruno formula~\ref{prop-faadibruno} to $\f = T(\dotp{\be}{e^{(\g-\C) / \epsilon}})$ yields
\begin{align}
  \f^{(\nu)}(x) &= \sum_{\lambda=1}^n T^{(\lambda)}(\dotp{\be}{e^{(\g-\C) / \epsilon}}) \sum_{p(\nu,\lambda)} (\nu!) \prod_{j=1}^n \frac{\big(\partial^{(l_j)}\dotp{\be}{e^{(\g-\C(x,.)) / \epsilon}} \big)^{k_j}}{(k_j!)(j!)^{k_j}}\label{eq-faa-pot-1}\\
  &\leq \sum_{\lambda=1}^n \bigg(\dotp{\be}{e^{(\g-\C) / \epsilon}}^{-\lambda} \sum_{k=1}^n \frac{(\epsilon)^\lambda P_{k-1}(\epsilon)}{(\inf\phi^{*\prime} + \epsilon\inf\phi^{*\prime\prime})^{2k - 1}}\bigg) \times\nonumber\\
  &\qquad\qquad\qquad\bigg( \dotp{\be}{e^{(\g-\C(x,.)) / \epsilon}}^{\sum k_j} Q_{\sum j k_j}(\tfrac{1}{\epsilon})\bigg)\label{eq-faa-pot-2}\\
%   \dotp{\be}{e^{(\g-\C) / \epsilon}}^{-\lambda} \sum_{k=1}^\lambda \frac{(\epsilon)^k P_{k-1}(\epsilon)\dotp{\be}{e^{(\g-\C(x,.)) / \epsilon}}^{\sum k_j} Q_{\sum j k_j}(\tfrac{1}{\epsilon})}{(\inf\phi^{*\prime} + \epsilon\inf\phi^{*\prime\prime})^{2\lambda - 1}}   \\
  &\leq \sum_{\lambda=1}^n \dotp{\be}{e^{(\g-\C) / \epsilon}}^{-\lambda} \sum_{k=1}^\lambda \frac{(\epsilon)^k P_{k-1}(\epsilon)\dotp{\be}{e^{(\g-\C(x,.)) / \epsilon}}^{\lambda} Q_{\nu}(\tfrac{1}{\epsilon})}{(\inf\phi^{*\prime} + \epsilon\inf\phi^{*\prime\prime})^{2k - 1}}   \label{eq-faa-pot-3}\\
  &\leq \sum_{\lambda=1}^n \sum_{k=1}^\lambda \frac{(\epsilon)^k P_{k-1}(\epsilon)}{(\inf\phi^{*\prime} + \epsilon\inf\phi^{*\prime\prime})^{2k - 1}} Q_{\nu}(\tfrac{1}{\epsilon}).\label{eq-bound-faa-pot}
\end{align}

Line~\eqref{eq-faa-pot-2} combines Inequalities~\eqref{eq-bound-faa-int} and~\eqref{eq-bound-faa-sink}.
%
The notation $P_\nu(\epsilon)$ represents a polynomial of order $\nu$ in $\epsilon$ whose coefficients depend on the norms $\norms{\phi^{*(k)} }_\infty$, and $Q_\nu(1 / \epsilon)$ represents a polynomial of order $\nu$ in $ 1 / \epsilon$ with no constant term and whose coefficients depend on the norms $\norms{\C^{(k)} }_\infty$.
%
Note that under Assumption~\ref{as:1}, $\phi^*$ is increasing and strictly convex on the compact $\Yy$, thus $\inf \phi^{*\prime} >0$ and $\inf \phi^{*\prime\prime} >0$. 
%
An important fact is that all terms $\dotp{\be}{e^{(\g-\C) / \epsilon}}$ disappear in the bound~\eqref{eq-bound-faa-pot}. Since all other contributions of this form disappear by bounding with $\norms{.}_\infty$, it means that $\norms{\f}_\infty$ is bounded independently of the mass of the input measure $\be$.

Thus the norm of the dual potential is bounded by 
\begin{align*}
  \norm{\f^{(\nu)}}_\infty &\leq \sum_{\lambda=1}^n \sum_{k=1}^\lambda \frac{\epsilon^k P_{k-1}(\epsilon)}{(\inf\phi^{*\prime} + \epsilon\inf\phi^{*\prime\prime})^{2k - 1}} Q_{\nu}(1/\epsilon).
\end{align*}
Again, note that the bound on the norm of $\f^{(\nu)}$ does not depend on $\dotp{\be}{e^{(\g-\C) / \epsilon}}$, thus it does not depend on the mass of the input measures $(\al,\be)$.

\hfill\break
\textbf{Proof of asymptotics.} 
%
Eventually we apply one last time the Faà Di Bruno formula~\ref{prop-faadibruno} to $h = \phi^*(-\f) + \epsilon\nabla\phi^*(-\f)$ with Inequality~\eqref{eq-bound-faa-pot} to get
\begin{align*}
  h^{(\nu)} &= \sum_{\lambda=1}^n (\phi^{*(\lambda)}(-\f) + \epsilon\phi^{*(\lambda+1)}(-\f)) \sum_{p(\nu,\lambda)} (\nu!) \prod_{j=1}^n \frac{(-\f^{(l_j)})^{k_j}}{(k_j!)(l_j!)^{k_j}},\\
  \norm{ h^{(\nu)} }_\infty &\leq \sum_{\lambda=1}^n \big(\norm{ \phi^{*(\lambda)} }_\infty + \epsilon\norm{ \phi^{*(\lambda+1)} }_\infty \big) \sum_{p(\nu,\lambda)} (\nu!) \prod_{j=1}^n \frac{\norm{ \f^{(l_j)} }_\infty^{k_j}}{(k_j!)(l_j!)^{k_j}}.
\end{align*}
Let's focus on the case $\epsilon\rightarrow 0$. In that case
\begin{gather*}
  \big(\norm{ \phi^{*(k)} }_\infty + \epsilon\norm{ \phi^{*(k+1)} }_\infty \big) \rightarrow \norm{ \phi^{*(k)} }_\infty,\\
  \frac{P_{k-1}(\epsilon)}{(\inf\phi^{*\prime} + \epsilon\inf\phi^{*\prime\prime})^{2k - 1}} \rightarrow cste,\\
  \epsilon^k Q_{\nu}(1/\epsilon) = O(1 / \epsilon^{\nu - k})
\end{gather*}
As for any $l_j$, $k$ varies from $1$ to $l_j$, we get that $\norms{\f^{(l_j)}}_\infty = O(1/ \epsilon^{l_j-1})$ and that the product of the norms is $O(1/ \epsilon^{|\nu| - \lambda})$. Since $\epsilon\rightarrow 0$, the principal term is given by the largest $|\nu|$ and smallest $\lambda$, i.e. $|\nu|=s$ and $\lambda=1$ (we are in $\Hh^s_{\al,\lambda}(\R^d)$). It gives that the Sobolev norm of $h$ is $O(1 / \epsilon^{s-1})$. Concerning the asymptotic $\epsilon\rightarrow\infty$, it gives
\begin{gather*}
  \frac{\epsilon^k P_{k-1}(\epsilon)}{(\inf\phi^{*\prime} + \epsilon\inf\phi^{*\prime\prime})^{2k - 1}} \rightarrow cste,\\
  \big(\norm{ \phi^{*(k)} }_\infty + \epsilon\norm{ \phi^{*(k+1)} }_\infty \big) Q_{\nu}(1/\epsilon) \rightarrow cste.
\end{gather*}
The second limit holds because $Q_\nu$ has no constant term. All in all, it gives that $\norms{ h^{(\nu)} }_\infty = O(1)$
\end{proof}

Now that the regularity of the dual potentials has been proved, we prove a bound on $|\OTb(\al,\be) - \OTb(\al_n,\be_n)|$ which allows to apply the PAC-framework results in RKHS.

\begin{proposition}\label{prop-ineq-ot-sob}
Assume that Assumption~\ref{as:1},\ref{as:2} hold. One has 
\begin{align}
  |\OTb(\al,\be) - \OTb(\al_n,\be_n)| &\leq\, 2 \sup_{\f\in\Hh^s_{\al,\lambda}(\R^d)} |\dotp{\al - \al_n}{\f}|
  +\, 2 \sup_{\f\in\Hh^s_{\be,\lambda}(\R^d)} |\dotp{\be - \be_n}{\g}|.\nonumber
\end{align}
\end{proposition}
\begin{proof}
Write as $\Aa(\al,\be,\f,\g)$ the functional optimized in the dual program~\eqref{eq-dual-unb}. The assumptions give that the optimal dual potentials exist, such that we write $\OTb(\al,\be) = \Aa(\al,\be,\f,\g)$ and $\OTb(\al_n,\be) = \Aa(\al_n,\be,\f_n,\g_n)$. The optimality of those potentials give the following suboptimality inequalities
\begin{align*}
  \Aa(\al,\be,\f_n,\g_n) - \Aa(\al_n,\be,\f_n,\g_n) 
  &\leq \Aa(\al,\be,\f,\g) - \Aa(\al_n,\be,\f_n,\g_n)\\
  &\leq \Aa(\al,\be,\f,\g) - \Aa(\al_n,\be,\f,\g).
\end{align*}
The central term is $\OTb(\al,\be) - \OTb(\al_n,\be)$, thus these bounds give
\begin{align*}
  |\OTb(\al,\be) - \OTb(\al_n,\be)| \leq &|\Aa(\al,\be,\f,\g) - \Aa(\al_n,\be,\f,\g)|\\
   &+ |\Aa(\al,\be,\f_n,\g_n) - \Aa(\al_n,\be,\f_n,\g_n)|.
\end{align*}
 We now bound each term. The proof is similar for both. Concerning the first term, one has
\begin{align*}
  |\Aa(\al,\be,\f,\g) &- \Aa(\al_n,\be,\f,\g)|
  = |\dotp{\al - \al_n}{-\phi^*(-\f)} - \epsilon\dotp{(\al - \al_n)\otimes\be}{e^{\frac{\f\oplus\g - \C}{\epsilon}} - 1}|.
\end{align*}
The measure $\al - \al_n$ has zero mean, thus constant terms cancel out. The dual optimality condition under Assumption~\ref{as:2} is $\dotp{\be}{\tefgc}= \nabla\phi^*(-\f)$. It yields
\begin{align*}
  |\Aa(\al,\be,\f,\g) - \Aa(\al_n,\be,\f,\g)|&= |\dotp{\al - \al_n}{-\phi^*(-\f) - \epsilon\nabla\phi^*(-\f)}|
  \leq \sup_{\f\in\Hh^s_{\al,\lambda}(\R^d)} |\dotp{\al - \al_n}{\f}|.
\end{align*}
Proposition~\ref{prop-dual-pot-sob} gives that $\phi^*(-\f) + \epsilon\nabla\phi^*(-\f)\in\Hh^s_{\al,\lambda}(\R^d)$, hence the last inequality with a supremum.

The proof is the same for the second term. The inequality for $|\OTb(\al,\be) - \OTb(\al_n,\be_n)|$ is obtained via a triangle inequality
\begin{align*}
  |\OTb(\al,\be) - \OTb(\al_n,\be_n)|\leq &|\OTb(\al,\be) - \OTb(\al_n,\be)|
   + |\OTb(\al_n,\be) - \OTb(\al_n,\be_n)|.
\end{align*}
The bound detailed previously applies for both terms, since it holds when one argument is fixed and the other is empirically estimated. 
\end{proof}