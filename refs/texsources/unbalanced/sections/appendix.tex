\section{Additional proofs}
\label{appendix-proofs}

\subsection{Proof of proposition~\ref{prop-nonexp}}
Assume $\epsilon=1$.
Take two pairs $(p_1,q_1)$, $(p_2,q_2)$ such that for $i\in\{1,2\}$, $q_i = \aprox{\phi^*}(p_i)$. This is equivalent to $e^{p_i-q_i}\in\partial\phi^*(q_i)$, and because $\partial\phi^*$ is a monotone operator one has
\begin{align*}
  (e^{p_1-q_1} - e^{p_2-q_2})(q_1 - q_2) \geq 0.
\end{align*}
Then one can use the first order convexity condition to get
\begin{align*}
  &e^{p_1-q_1} - e^{p_2-q_2} \geq e^{p_2-q_2}(p_1 - q_1 - p_2 + q_2),\\
  &e^{p_2-q_2} - e^{p_1-q_1} \geq e^{p_1-q_1}(p_2 - q_2 - p_1 + q_1),\\
  &\Rightarrow 0\geq (e^{p_1-q_1} - e^{p_2-q_2})(p_2 - q_2 - p_1 + q_1)\\
  &\Rightarrow (e^{p_1-q_1} - e^{p_2-q_2})(p_1 - p_2) \geq (e^{p_1-q_1} - e^{p_2-q_2})(q_1 - q_2)\geq 0.
\end{align*}
The case $e^{p_1-q_1} = e^{p_2-q_2}$ is trivial, and without loss of generality we can assume $e^{p_1-q_1} - e^{p_2-q_2} > 0$ by swapping indices if necessary. Eventually it gives the pointwise inequality
\begin{align*}
  |p_1-p_2| \geq |q_1 - q_2| = |\aprox{\phi^*}(p_1) - \aprox{\phi^*}(p_2)|.
\end{align*}
The above inequality gives that if $x\mapsto p(x)$ is a continuous function instead of a real number, then $x\mapsto\aprox{\phi^*}(p(x))$ is also a continuous function (Take $p_1 = p(x)$, $p_2 = p(y)$ and let $x\rightarrow y$). Now take $q_1 = \aprox{\phi^*}(\f)$ and $q_2 = \aprox{\phi^*}(\g)$ for some $(\f,\g)\in\Cc(\Xx)$. Since $\Xx$ is compact, suprema are attained and we can take the point $x\in\Xx$ such that
\begin{align*}
  \norm{q_1-q_2}_\infty = |q_1(x) - q_2(x)|\leq |\f(x) - \g(x)|\leq \norm{\f - \g}_\infty.
\end{align*}
This proves the statement for $\epsilon=1$. One has for any $\epsilon >0$
\begin{align*}
  \aprox{\phi^*}(p) = \epsilon\text{\upshape{Aprox}}_{(\phi/\epsilon)^*}^{1}(p/\epsilon).
\end{align*}
This relation allows to conclude for any $\epsilon$.


\subsection{Weak* continuity of $\OTb$}


\begin{theorem}[Convexity and continuity of $\OTb$]
%	\label{thm-continuity-unb}
	For any entropy $\phi$, $\OTb$ is convex on $\Mmp(\Xx)$ in $\al$ and $\be$ but not jointly convex.
	For any entropy $\phi$ such that Theorem~\ref{thm-cv-sink-compact} holds, consider a sequence $\al_n\rightharpoonup\al$ and $\be_n\rightharpoonup\be$ with $(\al,\be)\in\Mmp(\Xx)$, and write $(\f_n,\g_n)$ a sequence of optimal potentials for $\OTb(\al_n,\be_n)$. If $(\f_n,\g_n)$ can be uniformly bounded by a constant independent of $n$, then $\OTb(\al_n,\be_n)\rightarrow\OTb(\al,\be)$.
\end{theorem}
\begin{proof}
	With respect to convexity, $\OTb$ is a supremum of functions which are linear in $\al$ and linear in $\be$, but not jointly convex in $(\al,\be)$: it is convex in $\al$ \emph{and} in $\be$.
	%
	
	With respect to continuity, note that for any $n$, $(\f_n,\g_n)$ are $\gamma$-Lipschitz (Lemma~\ref{lem-smin-cost-regular}) and continuous on a compact set and are thus uniformly equicontinuous. Using the assumption that this sequence is uniformly bounded, the Ascoli-Arzelà Theorem allows us to show the relative compactness of the sequence in $\Cc(\Xx)\times\Cc(\Yy)$. Note that the Softmin and the aprox are $1$-Lipschitz (Lemma~\ref{lem-smin-lipschitz-func} and Proposition~\ref{prop-nonexp}) and the Softmin is weak* continuous in its input measure $\al$ or $\be$, thus for any converging subsequence $\f_{n_k}\rightarrow\f$ and $\g_{n_k}\rightarrow\g$, we get that $(\f,\g)$ is a fixed point of the Sinkhorn mapping for $(\al,\be)$ and is thus an optimal pair of potentials for $\OTb(\al,\be)$. Since the dual functional~\eqref{eq-dual-unb} is continuous in $(\al,\be,\f,\g)$, we get that for any subsequence $\OTb(\al_n,\be_n)\rightarrow\OTb(\al,\be)$, hence the continuity property.
\end{proof}

\begin{corollary}[Continuity of $\OTb$]\label{cor-continuity}
	Write $\al_n\rightharpoonup\al$ and $\be_n\rightharpoonup\be$ with $(\al,\be)\in\Mmpp(\Xx)$ such that for any $n$ there exists optimal dual potentials $(\f_n,\g_n)$. Then for any setting of Section~\ref{sec-exmp-f-div} we can uniformly bound this sequence and show that $\OTb$ is weak*-continuous.
\end{corollary}
\begin{proof}
	The case of strictly convex entropies is proved in Proposition~\ref{prop-uniform-conv}. In the balanced setting, potentials are defined up to a constant, thus we can assume without loss of generality that $\f_n(x^*)=0$ for some $x^*\in\Xx$. Because optimal potentials are $\gamma$-Lipschitz, we have that $\norm{\f_n}_\infty< \gamma\text{diam}(\Xx)$. Because the Sinkhorn update is 1-Lipschitz, we get that $\norm{\f_n}_\infty< 2\gamma\text{diam}(\Xx) + \epsilon|\log(m(\al_n))|$ and because $\al_n\rightharpoonup\al$ the mass term can be uniformly bounded, hence the result. When $\D_\phi=\rho\TV$ the aprox operator implies $\norm{\f_n}_\infty\leq\rho$ and $\norm{\g_n}_\infty\leq\rho$. In the case $\D_\phi=\RG_{[a,b]}$, we need to prove that for any $n$, at least one of the potentials $(\f_n,\g_n)$ is zero at some point of the support of $(\al_n,\be_n)$. If it is not the case, then we can replace $(\f_n,\g_n)$ by $(\f_n+ \lambda,\g_n-\lambda)$ with $\lambda\in\R$, and the expression of $\phi^*$ is such that the dual functional~\eqref{eq-dual-unb} is locally linear. Then we can locally increase the dual cost, which violates the optimality of $(\f_n,\g_n)$. Thus there exists $(x_n^*,y_n^*)$ such that $\f_n(x_n^*)=0$ or $\g_n(y_n^*)=0$, and we can derive a uniform bound similar to the balanced setting. Thus Theorem~\ref{thm-continuity-unb} holds and we get that $\OTb(\al_n,\be_n)\rightarrow\OTb(\al,\be)$.
\end{proof}


%%%%%%%%%%%%%%%%%%%%%%%%%%%%%%%%%%%%%%%%%%%%%%%%%%%%%%%%%%%%%%%%%%%%%%%%%%%%%%%%%%%%%%%%%%%%%%%%%%%%%
\subsection{Proof of Theorem~\ref{thm-diff-unb}}

The proof is mainly inspired from~\cite[Proposition 7.17]{santambrogio2015optimal}. 
Let us consider $\al$, $\delta\al$, $\be$, $\delta\be$
and $t$ in a neighborhood of $0$, as in Definition~\ref{def-diff-meas}.
We define the variation ratio $\tilde{\De}_t$
as $\tilde{\De}_t \eqdef \OTb(\al_t,\be_t) - \OTb(\al,\be) / t$.
%
%\paragraph{Weak$^*$ continuity.}
%
%As written in~\eqref{eq-dual-unb},  $\OTb(\al,\be)$ can be computed through a straightforward, \emph{continuous} expression of $(\f,\g)$.
%Combining this equation with Theorem~\ref{prop-uniform-conv} (that guarantees the \emph{uniform} convergence of potentials  for weakly converging sequences of probability measures) allows to conclude that $\OTb$ is weak* continuous when $(\al,\be)$ are not null.
%
%It is also convex in $\al$ and $\be$, thus the subgradient w.r.t $\al$ and $\be$ is always well defined. 
%%
%It remains to show that the subgradient is a gradient. Using the dual definition of $\OTb$
%and the continuity property of Proposition~\ref{prop-uniform-conv},
we provide lower and upper bounds on $\tilde{\De}_t$
as $t$ goes to $0$. The purpose of the proof is to show that the $\lim\sup$ and $\lim\inf$ coincide, proving the derivative to be well-defined.


%%%%%
\paragraph{Lower bound.}

First, let us remark that $(\f,\g)$ is a
\emph{suboptimal} pair of dual potentials for
$\OTb(\al_t,\be_t)$.
Hence, one has
\begin{align*}
&\OTb(\al_t,\be_t) \geqslant \dotp{\al_t}{-\phi^*(-\f)} + \dotp{\be_t}{-\phi^*(-\g)} -\epsilon\dotp{ \al_t\otimes\be_t}{ \efgc{\f\oplus\g}-1} \nonumber\\
&\OTb(\al,\be)=\dotp{\al}{-\phi^*(-\f)}+\dotp{\be}{-\phi^*(-\g)} -\epsilon\dotp{ \al\otimes\be}{ \efgc{\f\oplus\g}-1}.\\
&\tilde{\De}_t \geqslant
\dotp{\de\al}{-\phi^*(-\f)}
+
\dotp{\de\be}{-\phi^*(-\g)}
- \epsilon 
\langle \delta\al\otimes\be +
\al\otimes\delta\be , \efgc{\f\oplus\g}-1
\rangle+o(1).
\end{align*}

%%%%%
\paragraph{Upper bound.}

Conversely, let us denote the optimal pair 
of potentials for $\OTb(\al_t,\be_t)$ by $(\f_t,\g_t)$.
As $(\f_t,\g_t)$ are suboptimal potentials for $\OTb(\al,\be)$,
we get that
\begin{align*}
&\OTb(\al,\be)
\geqslant
\dotp{\al}{-\phi^*(-\f_t)}
+
\dotp{\be}{-\phi^*(-\g_t)} 
-
\epsilon\langle \al\otimes\be
, \efgc{\f_t\oplus\g_t}-1\rangle
\nonumber\\
 &\OTb(\al_t,\be_t) = \dotp{\al_t}{-\phi^*(-\f_t)}
+
\dotp{\be_t}{-\phi^*(-\g_t)}
 -\epsilon \langle\al_t\otimes\be_t, \efgc{\f_t\oplus\g_t}-1\rangle,\\
&\tilde{\De}_t \leqslant
\langle\delta\al,-\phi^*(-\f_t)\rangle
+
\langle\delta\be,-\phi^*(-\g_t)\rangle
- \epsilon 
\langle \delta\al\otimes\be_t +
\al_t\otimes\delta\be 
, \efgc{\f_t\oplus\g_t} \text{-}1
\rangle+o(1) \nonumber
\end{align*}

%%%%%
\paragraph{Conclusion.}

Now, let us remark that as $t$ goes to $0$, 
$\al+t\delta\al \rightharpoonup \al$ and $\be+t\delta\be \rightharpoonup \be$.
Using Proposition~\ref{prop-uniform-conv}, $\f_t$ and $\g_t$ converge uniformly
towards $\f$ and $\g$.
Combining the lower and upper bound, we get
\begin{align*}
\tilde{\De}_t \xrightarrow{t\rightarrow0}
&\langle\delta\al,-\phi^*(-\f)-\epsilon\dotp{\be}{\efgc{\f\oplus\g}-1}\rangle +
\langle\delta\be,-\phi^*(-\g)-\epsilon\dotp{\al}{\efgc{\f\oplus\g}-1}\rangle.
\end{align*}
One has $\nabla\phi^*(-\f)=\dotp{\be}{\efgc{\f\oplus\g}}$ and $\nabla\phi^*(-\g)=$ $\dotp{\al}{\efgc{\f\oplus\g}}$ when $\phi^*$ is differentiable.
It yields the last result.

%%%%%%%%%%%%%%%%%%%%%%%%%%%%%%%%%%%%%%%%%%%%%%%%%%%%%%%%%%%%%%%%%%%%%%%%%%%%%%%%%%%%%%%%%%%%%%%%%%%%%
\subsection{Proof of Theorem~\ref{thm-Feps_uniqueness}}

Write
$\E_\epsilon(\al,\mu)~\eqdef~\langle\al,\phi^*(-\epsilon\log \tfrac{\d\mu}{\d\al})\rangle
+\tfrac{\epsilon}{2}
\langle \mu, 
k_\epsilon\star \mu \rangle$ for $(\al,\mu)\in\Mmp(\Xx)\times\Mmp(\Xx)$.
Since $\C$ is bounded on the compact set $\Xx\times\Xx$ and $\al$
is a positive measure, we have that $\Fb(\al)\leq \E_\epsilon(\al,\al) <+\infty$.

\hfill\break
\textbf{Strict convexity and lower-semicontinuity of $E_\epsilon$ and $\Fb$.}
We use Equation~\eqref{eq-constrained-entropy-program} from Proposition~\ref{prop-entropy-reform}. It allows to add the constraint set $I = \{(\al,\mu)\in\Mmp(\Xx),\, \al\sim\mu\}$ which is jointly convex in $(\al,\mu)$. The function $\psi = \phi^* \circ (-\epsilon\log)$ is convex because both functions are convex and $\phi^*$ is nondecreasing. On the set $I$, $\al$ verifies $\al^\bot=0$ w.r.t. $\mu$, thus the term $\dotp{\al}{\phi^*\big(- \epsilon \log\big( \frac{\d\mu}{\d\al}\big)\,\big)}$ can be identified as a $\psi$-divergence (except it is not nonnegative) and is thus jointly convex in $(\al,\mu)$. The norm $\norm{.}_{k_\epsilon}$ is jointly convex, thus so is $E_\epsilon$. Eventually, we minimize a jointly convex function over a (jointly) convex set, and we get that $\Fb$ is convex. Since $\psi$-divergences are also l.s.c. then $E_\epsilon$ is also l.s.c.

\hfill\break
\textbf{Coercivity on $\mu$ and existence.}
Since $\Xx\times\Xx$ is compact and $k_\epsilon(x,y)>0$,
there exists $\eta > 0$ such that
$k(x,y)>\eta$ for all $x$ and $y$ in $\Xx$.
We thus get $\norm{\mu}_{k_\epsilon}^2 ~\geqslant~ \dotp{\mu}{1}^2\,\eta$. For $\mu\in\Mmp(\Xx)$ write $\mu_\al$ its restriction to $\text{spt}(\al)$. One has $m(\mu_\al)\geq m(\mu)$ and because $\phi^*\circ(-\epsilon\log)$ is nonincreasing one has thanks to Jensen inequality that
\begin{align*}
\E_\epsilon(\al,\mu) ~\geqslant~&  m(\al). \phi^*(-\epsilon\log(\frac{m(\mu_\al)}{m(\al)})) +\eta m(\mu)^2\\
 ~\geqslant~&  m(\al). \phi^*(-\epsilon\log(\frac{m(\mu)}{m(\al)})) +\eta m(\mu)^2\\
~\geqslant~& - m(\al)\epsilon\log(\frac{m(\mu)}{m(\al)}) +\eta m(\mu)^2.%\label{eq-coerc-entropy}
\end{align*}
Since $1 \in \text{dom}(\phi)$ one has $\phi^*(q) \geq q$. Thus whenever $m(\mu)$ goes to zero or infinity, $\E_\epsilon(\al,\mu)\rightarrow\infty$.
It allows to build a minimizing sequence $(\mu_n)$
for $\Fb(\al)$ such that
$m(\mu_n)$ is uniformly bounded by some constant
$M>0$.

The Banach-Alaoglu theorem holds and
asserts that $\{\, \mu\in\Mmp(\Xx) ~|~ m(\mu) \leqslant M \,\}$
is weakly compact;
we can extract a weakly converging subsequence
$\mu_{n_k}\rightharpoonup \mu_\infty$ from
the minimizing sequence $(\mu_n)$.
Since the map $\mu\mapsto\E_\epsilon(\al,\mu)$ is weakly
l.s.c.,
$\mu_\infty=\mu_\al$ realizes the minimum
of $\E_\epsilon$, proving the existence of minimizers.


\hfill\break
\textbf{Uniqueness.}
We assumed that the kernel $k_\epsilon$
is \emph{positive universal}.
The squared norm $\mu\mapsto\|\mu\|_{k_\epsilon}^2$
is thus a strictly convex functional, thus $\mu\mapsto \E_\epsilon(\al,\mu)$
is \emph{strictly} convex.
This ensures that $\mu_\al$ is uniquely defined.

\hfill\break
\textbf{Optimality of $\f$.}
If we consider the first order optimality in $\E_\epsilon$ we get $\al$-a.e. that
 $\tfrac{\d\mu_\al}{\d\al} k_\epsilon\star\mu_\al \in \partial\phi^*(-\epsilon\log \tfrac{\d\mu_\al}{\d\al}).$
Denoting $\f = \epsilon\log\tfrac{\d\mu_\al}{\d\al}$ this condition reads $e^{\f / \epsilon} \dotp{\al}{\efgc{\f}} \in \partial\phi^*(-\f)$.
Thus the potential $\f$ satisfies the optimality condition of the dual OT problem. The Radon-Nikodym-Lebesgue theorem only gives that $\f$ is $\al$-integrable, while we consider potentials in $\Cc(\Xx)$. Lemma~\ref{lem-smin-cost-regular} gives that $y \mapsto \Smin{\al}(\C(.,y) - \f)$ is continuous. The $\aprox{\phi^*}$ is Lipschitz thus continuous (Proposition~\ref{prop-nonexp}), so $\f=\Tt_\al(\f)$ is also continuous and optimal.

\hfill\break
\textbf{Strict convexity.}
We use Proposition~\ref{lem-injective-sym-pot} which gives that for two measures $\al\neq\be$ one has $\mu_\al\neq\mu_\be$ where both measure are the one attaining the optimal in $\Fb(\al)$ and $\Fb(\be)$. Since $k_\epsilon$ is universal the norm $\norm{.}_{k_\epsilon}$ is strictly convex when $\mu_\al\neq\mu_\be$. Write $\bar{\mu}$ the optimal measure for $t\al + (1-t)\be$ with $t\in(0,1)$. One has
\begin{align*}
\Fb(t\al + (1-t)\be) &= E_\epsilon(t\al + (1-t)\be, \bar{\mu})
\leq E_\epsilon(t\al + (1-t)\be, t\mu_\al + (1-t)\mu_\be)\\
&< t E_\epsilon(\al,\mu_\al) + (1-t) E_\epsilon(\be,\mu_\be)
< t \Fb(\al) + (1-t) \Fb(\be).
\end{align*}
Hence the strict convexity of the Sinkhorn entropy.


\hfill\break
\textbf{Continuity.} 
%Assuming that $\al_n\rightharpoonup\al$, we get that for $n$ large enough, there exists $\eta >0$ such that $0< m(\al) - \eta < m(\al_n) < m(\al) + \eta$.
%It allows to rewrite inequality~\eqref{eq-coerc-entropy} for $\al_n$. It reads
%  $\E_\epsilon(\al_n,\mu) ~\geqslant~ - (m(\al)+\eta)\epsilon\log(m(\mu)) + e^{-1}\epsilon  +\eta m(\mu)^2.$
%Here the log has been decomposed and the inequality $\forall x >0, \, x\log x \geq e^{-1}$ is used. Such inequality means that the functional is coercive independently of $n$. Thus one can assume that the sequence of measures $(\mu_n)$ that are optimal for each $(\al_n)$ have masses uniformly bounded, and apply the Banach-Alaoglu theorem again. One can extract a subsequence and by uniqueness of the optimizer $\mu$ for $\al$, the sequence $(\mu_n)$ necessarily weakly converges towards $\mu$. Eventually, one can apply such convergence of minimizers in $\E_\epsilon$ to prove the continuity of the Sinkhorn entropy.
The continuity is given by Theorem~\ref{thm-continuity-unb} and Corollary~\ref{cor-continuity} in the particular case $\al=\be$.

%%%%%%%%%%%%%%%%%%%%%%%%%%%%%%%%%%%%%%%%%%%%%%%%%%%%%%%%%%%%%%%%%%%%%%%%%%%%%%%%%%%%%%%%%%%%%%%%%%%%%
%\subsection{Proof of Lemma~\ref{lem-inf-strict-cvx}}
%\todo{REMOVE}
%\begin{lemma}\label{lem-inf-strict-cvx}
%Let $\Cc, \Xx$ be two convex non-empty sets, and $\f:\Xx\times\Cc\rightarrow\R$ be a function. We define for any $x\in\Xx$, $\g(x) = \inf_{y\in\Cc} f(x,y)$.
%If $\f$ is strictly convex in $(x,y)$ and attains its minimum for any $x\in\Xx$, then $\g$ is also strictly convex.
%\end{lemma}
%\begin{proof}
%Let us take $x_0\neq x_1 \in\Xx$. since the infimum is attained, there exists $(y_0, y_1)\in\Cc$ such that for $t\in (0,1)$
%\begin{align*}
%  (1-t) \g(x_0) + t \g(x_1) &= (1-t) \f(x_0, y_0) + t \f(x_1, y_1)\\
%  & > \f((1-t) x_0 + t x_1, (1-t) y_0 + t y_1 )\\
%  & > \g((1-t) x_0 + t x_1).
%\end{align*}
%\end{proof}


%%%%%%%%%%%%%%%%%%%%%%%%%%%%%%%%%%%%%%%%%%%%%%%%%%%%%%%%%%%%%%%%%%%%%%%%%%%%%%%%%%%%%%%%%%%%%%%%%%%%%%%
\subsection{Proof of Proposition~\ref{prop-Seps-ineq-norm}}

In the latter development we identify symetric terms with a kernel norm through 
$\Vert \be e^{\frac{\g_{\be}}{\epsilon}}\Vert^2_{k_\epsilon} = \dotp{\be\otimes\be}{e^{\frac{(\g_{\be}\oplus \g_{\be} - \C )}{\epsilon}}}.$

Under our assumptions, we know from Theorem~\ref{thm-Feps_uniqueness} that $\f_\al$ and $\g_\be$ exist and are unique. The idea of the proof is to say that the pair of potentials $(\f_{\al}, \g_{\be})$ is suboptimal for $\OTb(\al,\be)$. Since its definition is a supremum over $\Cc(\Xx)$ we get a lower bound that gives
\begin{align*}
 & \OTb(\al,\be)  \geq 
  		- \dotp{\al}{\phi^*(-\f_\al )} -  \dotp{\be}{\phi^*(-\g_\be)}
		-\epsilon \dotp{\al\otimes\be}{e^{\frac{(\f_\al\oplus \g_\be - \C )}{\epsilon}} - 1} \\
%%%%
	&\qquad \geq - \dotp{\al}{\phi^*(-\f_\al )} - \dotp{\be}{\phi^*(-\g_\be )}
		-\epsilon \dotp{\al\otimes\be}{e^{\frac{(\f_\al\oplus \g_\be - \C )}{\epsilon}}} 
		+ \epsilon m(\al)m(\be)\nonumber\\[0.4em]
%%%%
& \qquad \geq - \dotp{\al}{\phi^*(-\f_\al)} - \dotp{\be}{\phi^*(-\g_\be)} -\epsilon \dotp{\al\otimes\be}{e^{\frac{(\f_\al\oplus \g_\be - \C )}{\epsilon}}} \\
%
&\quad\qquad-\tfrac{\epsilon}{2}( \Vert \al e^{\frac{\f_\al}{\epsilon}}\Vert^2_{k_\epsilon} - m(\al)^2 + \Vert \be e^{\frac{\g_\be}{\epsilon}}\Vert^2_{k_\epsilon} - m(\be)^2)\\
%
&\quad \qquad+\tfrac{\epsilon}{2}( \Vert \al e^{\frac{\f_\al}{\epsilon}}\Vert^2_{k_\epsilon} - m(\al)^2 + \Vert \be e^{\frac{\g_\be}{\epsilon}}\Vert^2_{k_\epsilon} - m(\be)^2) + \epsilon m(\al)m(\be)\\ 
%%%%
& \qquad \geq - \dotp{\al}{\phi^*(-\f_\al)} - \tfrac{\epsilon}{2} (\Vert \al e^{\frac{\f_\al}{\epsilon}}\Vert^2_{k_\epsilon} - m(\al)^2) 
	- \dotp{\be}{\phi^*(-\g_\be)} - \tfrac{\epsilon}{2}(\Vert \be e^{\frac{\g_\be}{\epsilon}}\Vert^2_{k_\epsilon} - m(\be)^2)\nonumber \\
%
&\quad \qquad +\tfrac{\epsilon}{2}\big( \Vert \al e^{\frac{\f_\al}{\epsilon}}\Vert^2_{k_\epsilon} + \Vert \be e^{\frac{\g_\be}{\epsilon}}\Vert^2_{k_\epsilon}-2 \dotp{\al\otimes\be}{e^{\frac{(\f_\al\oplus \g_\be - \C )}{\epsilon}}}\big)\\
&\quad \qquad	 + \epsilon m(\al)m(\be) - \tfrac{\epsilon}{2}(m(\al)^2 + m(\be)^2) \\
%%%
& \qquad  \geq \tfrac{1}{2}\OTb(\al,\al)+  \tfrac{1}{2}\OTb(\be,\be) 
	+ \tfrac{\epsilon}{2} \Vert \al e^{\frac{\f_\al}{\epsilon}} 
	- \be e^{\frac{\g_\be}{\epsilon}}\Vert^2_{k_\epsilon} 
	- \tfrac{\epsilon}{2} (m(\al) - m(\be))^2.
\end{align*}
With the last line we deduce the desired bound from the definition of $\Sb$.